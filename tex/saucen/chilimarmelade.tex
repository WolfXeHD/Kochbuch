\begin{otherlanguage}{ngerman}

\setHeadlines
{% translation
    inghead = Zutaten,
    prephead = Zubereitung,
    hinthead = Tipp,
    continuationhead = Fortsetzung,
    continuationfoot = Fortsetzung auf n\"achster Seite,
    portionvalue = Personen,
}

\begin{recipe}
[ % Optionale Eingaben
    preparationtime = {\unit[2]{Stunden}},
    portion = \portion{8},
    % bakingtime = {\unit[1]{Stunde}},
    % bakingtemperature={\protect\bakingtemperature{
    %     topbottomheat=\unit[190]{\textcelcius}
    %     }},
    portion = {\portion{8-10}},
    source = online
]
{Chili-Marmelade}
    \graph
    {% Bilder
        % small=pic/glass,    % kleines Bild
        % big=pic/ingredients % großes (längeres) Bild
    }
    \ingredients
    {% Zutaten
        \unit[1000]{g} & Paprika (rot \& gelb) \\
        \unit[500]{g} & Gelierzucker (2:1) \\
        \unit[2] & gro{\ss}e Chili \\
        \unit[350]{ml} & Apfelessig \\
    }

    \preparation{%
    \step 750g der Paprika p\"urieren.
    \step Chilis und 250g der Paprika fein schneiden und Kerne entfernen.
    \step P\"urierte Paprika, Apfelessig, Chili und Gelierzucker vermischen und in einem Topf zum kochen bringen und 10 Minuten sprudelnd kochen lassen.
    \step Die fein geschnittene Paprika hinzugeben, umr\"uhren und in ausgekochte Gl\"aser abf\"ullen.

    }

    \hint
    {% Tipp
        Eignet sich hervorragend zu K\"ase und Fleisch.
    }

\end{recipe}

\end{otherlanguage}
