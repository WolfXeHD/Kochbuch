%% Mousse au Chocolat Example

\begin{otherlanguage}{ngerman}

\setHeadlines
{% translation
    inghead = Zutaten,
    prephead = Zubereitung,
    hinthead = Tipp,
    continuationhead = Fortsetzung,
    continuationfoot = Fortsetzung auf n\"achster Seite,
    portionvalue = Personen,
}

\begin{recipe}
[ % Optionale Eingaben
    preparationtime = {\unit[2]{Stunden}},
    portion = \portion{8},
    bakingtime = {\unit[1]{Stunde}},
    bakingtemperature={\protect\bakingtemperature{
        topbottomheat=\unit[190]{�C}
        }},
    portion = {\portion{8-10}},
    source = Erna
]
{K\"asekuchen}
    \graph
    {% Bilder
        % small=pic/glass,    % kleines Bild
        % big=pic/ingredients % gro�es (l�ngeres) Bild
    }
    \ingredients
    {% Zutaten
        % 2 Tafeln & dunkle Schokolade (�ber \unit[70]{\%})\\
        % 3 & Eier\\
        % \unit[200]{ml} & Sahne\\
        % \unit[40]{g} & Zucker\\
        % \unit[50]{g} & Butter
        \unit[350]{g} & Zucker \\
        \unit[200]{g} & Magarine \\
        \unit[6] & Ei \\
        \unit[250]{g} &  Mehl \\
        \unit[250]{ml} &  Sonnenblumen\"ol \\
        \unit[750]{g} & Quark \\
        \unit[5]{EL} & Milch \\
        \unit[1]{} & Zitrone \\
        \unit[1]{} & Vanillepudding- Pulver \\
        \unit[1]{} & Vanillezucker \\
    }

    \preparation{%
    \step Zun�chst wird der Boden zubereiten. Die vorgesehe- ne Menge reicht f�r 3 B�den aus. Es werden 100g
Zucker mit 200g Magarine schaumig ger�hrt.
    \step 1 Ei hinzuf�gen und 250g Mehl unterr�hren, sodass eine feste Teigmasse entsteht und den Teig mindes-
tens eine Stunde im K�hlschrank ruhen lassen.
    \step In einer weiteren Sch�ssel werden 750g Quark und 5 Eier verr�hrt.
    \step Die restlichen 250g Zucker werden nach und nach hin- zugegeben. Nun das �l langsam hinzugeben.
    \step Den Saft einer Zitrone, die Milch, das Puddingpulver, den Vanillezucker, sowie eine Prise Salz hinzugeben
und den K�se glattr�hren.
    \step Den Boden in eine gut eingefettete Springform geben und den K�se daraufgeben.
    \step Den Kuchen eine Stunde im Ofen bei \unit[190]{�C} backen.

    }

    \hint
    {% Tipp
        Falls der Kuchen zu dunkel wird, kann man ihn mit Alufolie abdecken.
    }

\end{recipe}

\end{otherlanguage}
