% Complete recipe example
\begin{recipe}
[%
    preparationtime = {\unit[15]{min}},
    bakingtime={\unit[30]{min}},
    bakingtemperature={\protect\bakingtemperature{
        % fanoven=\unit[230]{\textcelcius},
        topbottomheat=\unit[180]{\textcelcius},
        % topheat=\unit[195]{°C},
        % gasstove=Level 2
        }},
    portion = {\portion{6}},
    % calory={\unit[3]{kJ}},
    source = {Soraya}
]
{Brownies de Milo}

    \ingredients{%
        \unit[\nicefrac{3}{4}]{taza} & Azucar\\
        \unit[\nicefrac{3}{4}]{taza} & Milo\\
        \unit[\nicefrac{3}{4}]{taza} & Harina\\
        \unit[\nicefrac{1}{2}]{taza} + 2 cdas & Mantequilla derretida\\
        \unit[\nicefrac{1}{4}]{cdita} & Polvo de hornear\\
        \unit[2]{cditas} & esencia de vainilla\\
        2  & Huevos\\
        1  & Yema\\
        % \unit[2]{tasas} & Harina\\
    }

    \preparation{%
        \step Precalentar el horno a 180\textcelcius.
        \step Mezclar muy bien la mantequilla con el az\'ucar, a\~nadir lentamente los huevos revolviendo constantemente. A\~nadir la vainilla y el Milo y revolver.
        \step Mezclar la harina con el polvo de hornear y a\~nadir esta mezcla a la primera.
        \step En un molde/refractaria grande engrasado y enharinado verter la mezcla y llevar al horno por 30 minutos a 180\textcelcius.
    }

    \hint{%
    \begin{itemize}
        \item La yema es indispensable para que queden melcochudos.
        \item Dejar enfriar antes de cortar.
    \end{itemize}
    }

\end{recipe}
