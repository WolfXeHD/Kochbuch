% Complete recipe example
\begin{recipe}
[%
    preparationtime = {\unit[15]{min}},
    bakingtime={\unit[12-15]{min}},
    bakingtemperature={\protect\bakingtemperature{
        % fanoven=\unit[230]{\textcelcius},
        topbottomheat=\unit[230]{\textcelcius},
        % topheat=\unit[195]{\degree C},
        % gasstove=Level 2
        }},
    portion = {\portion{4}},
    % calory={\unit[3]{kJ}},
    source = {Erna}
]
{Masa de Pizza}

    % \graph
    % {% pictures
    %     small=pic/glass,     % small picture
    %     big=pic/ingredients  % big picture
    % }

    % \introduction{%
    % \blindtext
    % }

    \ingredients{%
        \unit[500]{g} & Harina\\
        \unit[4]{g} & Levadura\\
        \unit[\nicefrac{1}{2}]{cda} & Azucar\\
        \unit[\nicefrac{1}{4}]{cdita} & Sal\\
        \unit[\nicefrac{1}{4}]{taza} & Leche\\
        \unit[300]{ml} & Agua tibia\\
        \unit[2]{cda} & Aceite de oliva\\
        % \unit[2]{tasas} & Harina\\
    }

    \preparation{%
        \step Poner todos los ingredientes en un recipiente y mezclarlos por al menos 5 minutos. Dejar reposar la masa por aproximadamente 30 minutos.
        \step Dividir la masa en porciones (2-6) y hacer bolas con cada pedazo
        \step Poner la pizza en el horno (a 220\textcelcius, por aprox. 15-20 minutos), en una piedra para pizza (a 250\textcelcius, por aprox. 10-15 minutos) o en una sart\'en el\'ectrica (tercer nivel, primer lado 3 minutos, despu\'es voltear y agregar los ingredientes y otros 3 minutos).
    }

    % \suggestion[Headline]
    % {%
    %     \blindtext
    % }

    % \suggestion{%
    %     \blindtext
    % }

    \hint{%

    }

\end{recipe}
